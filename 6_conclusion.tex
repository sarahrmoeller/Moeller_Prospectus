\chapter{Conclusion}
\label{chap:conclusion}

The studies proposed here will investigate the potential to advance the science of linguistics through the integration of NLP machine learning systems. This research has broad intellectual merit and potential impacts. It will provide more usable data in linguistics, leverage field data for NLP models, and test the models efficacy with limited and noisy data which is often the only data available for endangered languages. It also advances murky issues surrounding the nature and role of syntactic categories with an empirical study.

The outcomes of this research will have broader impact on future direction of language documentation and description methods and workflow. If theory-based segmentation choices negatively affect the performance of machine learning models on morpheme segmentation and glossing, then linguists who wish to benefit from automated assistance in this task will have to adjust their choices and their investment in manual annotation consistency. If the annotation quality as well as the amount of data determines how different machine learning models perform, then linguists may need to educate themselves on the difference between machine learning models. If these issues make little difference, then documentary and descriptive linguists can continue with currently accepted methods and workflows.

If documentary and descriptive IGT that has been manually annotated can be utilized for computational induction of morphological inflection patterns with minimal need for domain experts, then both linguists and NLP practitioners can confidently increase their collaboration. They can further develop NLP for low-resource settings and perhaps focus together on systems that mutually benefits the discipline of linguistics, the development of NLP, and the practical needs of communities engaged in language revitalization or maintenance. 

If the presence of POS tags makes a significant difference in the results of other automated tasks that are helpful to the documentary and descriptive workflow, then linguists may want to re-examine their accepted workflow and emphasize early analysis and labeling of syntactic categories. If the difference is trivial, then linguists have little reason to allocate early time and funding to POS tagging despite its importance to NLP. NLP practitioners, in turn, may need to reconsider their reliance on POS tags for some tasks. 

A natural future direction is to address the practical implications of integrating machine learning. One way to address this is to ``active learning'' experiments where machine learning models are iteratively re-trained with a small set of new manual annotations. A key issue is how to select the data for new annotations so that the model will improve most quickly with least manual effort. Selection methods based on computational and linguistic motivations could be tested and compared. For example, active learning for paradigm induction would involve first training on partial tables, then comparing results after retraining on data selected by different methods: 1) selecting word forms from the model’s predictions that it had least ``confidence'' about, 2) randomly selecting word forms, 3) selecting a random number word forms but making sure they are evenly distributed among tables, or 4) selecting word forms so that each table has the same minimum number of forms per table. This same experiment could be conducted for joint segmentation and glossing. The first two methods would remain the same, but other methods could be 3) selecting a number of least frequent word types, 4) selecting word forms based on statistically rare letter sequence combinations (in attempt to identify rare morphological combinations), or some other method motivated by linguistic facts of the language such as 5) annotating only nouns in Lezgi which are most likely to have irregular allomorphy. Either of these experiments are potential additional or alternative studies to the research that is described in this Prospectus.
